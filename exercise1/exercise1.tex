\documentclass{article}
\usepackage{amsmath}

\title{Exercise 1}
\author{Daniel Gallo}
\date{September 2020}

\begin{document}
\maketitle
\section*{18.8}
Let $Y$ be an ordered set in the order topology. Let $f,\ g \colon X \rightarrow Y$ be continuous.
\begin{enumerate}
    \item Show that the set $\left\{x \mid f(x) \leq g(x)\right\}$ is closed in X.
    \item Let $h\colon X \rightarrow Y$ be the function
    \begin{equation*}
        h(x) = min\left\{f(x),\ g(x)\right\}
    \end{equation*}
    Show that $h$ is continuous.
\end{enumerate}
\subsection*{Every order topology is Hausdorff}
First of all, we will proof that $Y$ is indeed Hausdorff. Given two elements $a,\ b \in Y$ such that $a < b$, we can consider the following set.
\begin{equation*}
    S = \left\{x \mid a < x < b\right\}
\end{equation*}
Now the following situations arise.
\begin{itemize}
    \item If S is empty, we can choose $U = \left(-\infty,\ b\right)$ and $V = \left(a,\ \infty\right)$.
    \item If there is at least one element (which we can call $c$), we can choose $U = \left(-\infty,\ c\right)$ and $V = \left(c,\ \infty\right)$.
\end{itemize}
In either case, $U$ is a neighborhood of $a$ and $V$ is a neighborhood of $b$, since open rays are open. Furthermore, their intersection is empty. We can conclude then that $Y$ is Hausdorff.
\section*{19.3}
If each $X_\alpha$ is a Hausdorff space, then $\prod X_\alpha$ is a Hausdorff space in both the box and product topologies.
\end{document}